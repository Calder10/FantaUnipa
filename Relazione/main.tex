\documentclass[12pt,a4paper]{article}
\usepackage[utf8]{inputenc}
\usepackage[italian]{babel}
\usepackage{amsmath}
\usepackage{amsfonts}
\usepackage{amssymb}
\usepackage{graphicx}
\usepackage[left=2cm,right=2cm,top=2cm,bottom=2cm]{geometry}
\title{FantaUnipa}
\author{Salvatore Calderaro \and Gaspare Casano}
\usepackage{hyperref}
\begin{document}
\maketitle
\newpage
\tableofcontents
\newpage
\begin{abstract}
FantaUnipa è un software creato per simulare il gioco del fantacalcio. Esso è stato sviluppato mediante l'utilizzo di diversi design pattern.
\end{abstract}
\newpage
\section{Introduzione}
FantaUnipa è un software che permette la simulazione di un torneo del famoso gioco del fantacalcio. Il fantacalcio è un gioco che consiste nel creare e gestire squadre virtuali formate da calciatori reali, scelti fra quelli che giocano il campionato a cui il gioco si riferisce (ad esempio Serie A, Premier League, etc). Il software, dopo che l'utente ha effettuato la registrazione al sistema, crea gli altri cinque utenti virtuali che parteciperanno al torneo. La creazione degli utenti (virtuali e non) è stata implementata mediante l'utilizzo del design pattern strutturale \textit{Builder}. Questa scelta è giustificata dal fatto che,essendo la classe Fantallenatore formata da diversi attributi, il suo metodo costruttore avrebbe avuto troppi attbuti passati a parametro; l'utilizzo del Builder ci ha permesso di costruire l'oggetto tramite l'utilizzo di appositi metodi per ogni singolo attributo.\\
Dopo avere effettuato la registrazione, l'utente dovra inserire il nome e il logo della sua fantasquadra e parallelamente viene effettuata la stessa operazione per gli utenti virtuali. Una volta che tutti gli utenti hanno scelto il nome e il logo della loro fantasquadra inizia l'asta per l'acquisto dei giocatori. Ogni fantallenatore deve acquistare:
\begin{itemize}
\item 3 portieri;
\item 8 difensori;
\item 8 centrocampisti;
\item 6 attaccanti.
\end{itemize}
Durante le diverse sessioni dell'asta ogni partecipante può puntare, rilanciare o rinuciare a quel determinato giocatore.\\
L'asta è stata implementata mediante l'utilizzo dei design pattern comportamentali \textit{Observer} e \textit{Strategy}. In particolare Observer è stato utilizzato per notificare a tutti i fantallenatori (gli observer) il cambiamento di stato (valore della puntata, o un'eventuale rinuncia) dell'oggetto osservabile (sessione di asta per un determinato calciatore). In questo modo, in ogni momento, ogni fantallenatore è a conoscenza dello stato dell'asta e può decidere se rilanciare o abbandonare l'asta; strategy è stato utilizzato per implementare diverse strategie di rilancio utilizzate dagli utenti virtuali. In particolare è stato possibile cambiare a runtime il tipo di algoritmo da utilizzare in base alla strategia che ha deciso di adottare l'utente virtuale.\\
Dopo che tutti gli utenti hanno completato la creazione della propria rosa, il software genera il torneo composto da sei squadre che si sfideranno secondo la modalità del girone all'italiana. Per la creazione del torneo è stato utilizzato il pattern creazionale Singleton ed il pattern strutturale Facade. Il pattern \textit{Singleton} è stato utilizzato per rendere l'entità torneo istanziabile una e una sola volta; \textit{Facade} è stato utilizzato per nascondere ai fantallenatori la  complessita dell'entità torneo e per rendere più facile l'utilizzo dei sottosistemi da cui esso è composto.\\
Una volta creato il torneo, per ogni giornata, l'utente dovrà inserire la propria formazione scegliendo fra quattro diversi moduli disponibili (4-4-2, 4-3-3, 3-4-4, 3-5-2); la creazione della formazione per tutti gli utenti è stato implementato utilizzando il design pattern creazionale \textit{Factory method} il quale ci ha permesso di creare una classe (FormazioneFactory) che in base al tipo di modulo scelto dall'utente delega la creazione dell'oggetto alla sottoclasse corrispondente al modulo scelto.\\
Alla fine di ogni giornata l'utente avrà accesso ad una serie di informazioni inerenti il torneo (ad esempio classifica, risultati delle partite giocate etc).\\ Quando si conclude il torneo verranno visualizzate le squadre che si sono classificate nelle prime tre posizioni.
\section{Analisi e specifica dei requisiti}
\subsection{Stakeholder principali}
L'unico stakeholder del software FantaUnipa è il fantallenatore, che è colui che utilizza il software per simulare un torneo di fantacalcio. 
\subsection{Specifica dei requisiti funzionali}
Di seguito viene riportato l'elenco dei requisiti funzionali del fantallenatore:
\begin{itemize}
\item registrarsi al sistema;
\item effettuare il login;
\item creare la propria fantasquadra;
\item partecipare all'asta per l'acquisto dei giocatori;
\item schierare la formazione;
\item visualizzare il calendario del torneo;
\item visualizzare la propria rosa;
\item visualizzare la rosa degli altri fantallenatori;
\item visualizzare i voti e risultati delle partite precedenti;
\item visualizzare la classifica del torneo.
\end{itemize}
\subsection{Specifica dei requisiti NON funzionali}
Di seguito è riportata l'elenco dei requisiti non funzionali del sistema:
\begin{itemize}
\item usabilità: l'interfaccia utente è stata implementata cercando di garantire la massima usabilità ed una facile localizzazione dei comandi da utilizzare;
\item velocità: il software è stato progettato in modo tale da avere tempi di risposta brevi fra un'operazione ed un'altra;
\item portabilità: il software può essere eseguito su diversi sistemi;
\end{itemize}
\subsection{Casi d'uso}
\subsubsection{Registrazione al sistema}
Attore principale: \textbf{Fantallenatore}\\
\newline
\textbf{PASSI}
\begin{enumerate}
\item clicca sul bottone registrati;
\item il sistema visualizza il modulo per la registrazione;
\item l'utente compila tutti campi e preme il tasto registrati;
\item il sistema acquisisce e controlla i dati;
\item il sistema mostra un messaggio di conferma e visualizza il modulo di login.
\end{enumerate}
\textbf{ESTENSIONI}
\begin{enumerate}
\item uno o più campi vuoti: in questo caso viene mostrato all'utente un messaggio di errore e il sistema ritorna al passo 2;
\item la password non rispetta gli standard definiti dal sistema: in questo verrà mostrato all'utente un messaggio di errore e il sistema ritorna al passo 2;
\item l'username è stato già utilizzato: in questo caso verrà mostrato all'utente un messaggio di errore e il sistema ritorna al passo 2;
\end{enumerate}
\subsubsection{Effettuare il login}
Attore principale: \textbf{Fantallenatore}\\
\newline
\textbf{PASSI}
\begin{enumerate}
\item il sistema mostra all'utente il modulo per effettuare il login;
\item l'utente compila il modulo e clicca sul bottone login;
\item il sistema acquisisce e controlla i dati;
\item il sistema visualizza la schermata home del software nel caso in cui il fantallenatore avesse già creato la sua squadra, altrimenti viene visualizzata la schermata per la creazione della squadra.
\end{enumerate}
\textbf{ESTENSIONI}
\begin{itemize}
\item campi non validi: in questo caso il sistema visualizza un messaggio d'errore e torna al punto 1.
\end{itemize}
\subsubsection{Creazione squadra}
Attore principale: \textbf{Fantallenatore}\\
\newline
\textbf{PASSI}
\begin{enumerate}
\item il sistema visualizza il modulo per la creazione della squadra;
\item l'utente compila il modulo inserendo il nome della squadra ed il logo;
\item il sistema acquisisce i dati;
\item il sistema visualizza la finestra che permetterà all'utente di iniziare l'asta per la creazione della rosa.
\end{enumerate}
\textbf{ESTENSIONI}
\begin{itemize}
\item campi vuoti: il sistema visualizza un messaggio di errore e torna al punto 1.
\end{itemize}
\subsubsection*{Partecipazione all'asta}
Attore principale: \textbf{Fantallenatore}\\
\newline
\textbf{PASSI}
\begin{enumerate}
\item il sistema visualizza una finestra che consente all'utente di ricercare i giocatori che intende acquistare
\item l'utente  a questo punto può:
\begin{itemize}
\item inserire il nome del giocatore nell'apposito campo e premere il bottone cerca: in questo caso gli verrà mostrato il nome del giocatore e il bottone scegli
\item premere il bottone Mostra Tutti e cliccare su uno dei giocatori della lista.
\end{itemize}
\item in entrambi i casi precedenti dopo il click il sistema visualizzerà una finestra in cui si svolgeranno le operazioni per l'asta;
\item l'utente inserisce la puntata e clicca su punta;
\item il sistema visualizza le puntate degli altri giocatori;
\item l'utente può decidere se rinunciare cliccando sul bottone rinuncia o se effettuare una nuova puntata.
\end{enumerate}
\textbf{ESTENSIONI}
\begin{itemize}
\item il giocatore scelto è stato già acquistato oppure l'utente non ha inserito il nome di alcun giocatore: in questo caso viene mostrata una finestra di notifica e il sistema ritorna la punto 1;
\item puntata non consentita: il sistema mostra una finestra di errore e ritorna al punto 
\end{itemize}
\end{document}